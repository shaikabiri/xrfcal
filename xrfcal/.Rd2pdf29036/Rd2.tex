\documentclass[a4paper]{book}
\usepackage[times,inconsolata,hyper]{Rd}
\usepackage{makeidx}
\usepackage[utf8]{inputenc} % @SET ENCODING@
% \usepackage{graphicx} % @USE GRAPHICX@
\makeindex{}
\begin{document}
\chapter*{}
\begin{center}
{\textbf{\huge Package `xrfcal'}}
\par\bigskip{\large \today}
\end{center}
\inputencoding{utf8}
\ifthenelse{\boolean{Rd@use@hyper}}{\hypersetup{pdftitle = {xrfcal: Calibrating XRF scanners elemental counts to reference elemental concentrations}}}{}\begin{description}
\raggedright{}
\item[Title]\AsIs{Calibrating XRF scanners elemental counts to reference elemental concentrations}
\item[Version]\AsIs{0.0.0.9000}
\item[Description]\AsIs{Methods for converting elemental counts acquired by XRF scanner to reference elemental concentrations acquired by conventional methods such as ICP (Kabiri et al. (2022) <doi>, Weltje and Tjalingii (2008) <}\Rhref{https://doi.org/10.1016/j.epsl.2008.07.054}{doi:10.1016/j.epsl.2008.07.054}\AsIs{>).}
\item[License]\AsIs{MIT + file LICENSE}
\item[Encoding]\AsIs{UTF-8}
\item[Roxygen]\AsIs{list(markdown = TRUE)}
\item[RoxygenNote]\AsIs{7.2.1}
\item[Depends]\AsIs{R (>= 2.10)}
\item[LazyData]\AsIs{true}
\item[Imports]\AsIs{lmodel2,
zCompositions,
agrmt,
stats,
caret,
Cubist,
randomForest,
tibble,
Rdpack}
\item[RdMacros]\AsIs{Rdpack}
\end{description}
\Rdcontents{\R{} topics documented:}
